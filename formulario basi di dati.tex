\documentclass[a4paper,12pt]{article}

\usepackage[T1]{fontenc}
\usepackage[utf8]{inputenc}
\usepackage[italian]{babel}
\usepackage{eucal,enumitem}
\usepackage{graphicx}
\usepackage{amsmath,amssymb,amsthm}
\usepackage{tikz}
\usepackage{cancel} 
\usepackage{multicol}
\usepackage{color}
\usepackage{fancyhdr}
\usepackage{amsfonts}
\usepackage{amssymb}

\newcommand{\Anegato}{\overline{\text{A}}}
\newcommand{\Bnegato}{\overline{\text{B}}}
\newcommand{\modificato}{\underset{\sim}}


\begin{document}
	\large{\title{FORMULE DI BASI DI DATI}}
	\maketitle
	\begin{titlepage}
		\title{{\huge \textbf{FORMULE SUI B-TREE}}}
		\maketitle\\\\
		DEFINIZIONE DI B-TREE :	\\
		Un B-TREE è un albero bilanciato, organizzato a nodi, ogni nodo corrisponde ad un blocco dati di uno storage device \\ \\
		IMPORTANTE : \\
		T(g,h) è un albero bilanciato di ordine g e altezza h. Le cardinalità/configurazioni possibili sono rappresentate nella tabella sottostante.
		Ricordiamo che : |sk| = numero di chiavi selezionate per nodo(nodo != radice). \\ \\
		\underline{Numero Massimo e minimo di chiavi per ogni nodo o radice:} \\ \\ 
		\begin{tabular}{|c|c|c|}
			\hline
			CHIAVI & MIN & MAX \\
			\hline
			Radice & 1 & 2g \\
			\hline
			Nodo & g & 2g\\
			\hline
		\end{tabular} \\ \\
		\underline{Numero Massimo e minimo di figli per ogni nodo o radice:} \\ \\ 
		\begin{tabular}{|c|c|c|}
			\hline
			FIGLI & MIN & MAX \\
			\hline
			Radice & 0 & 2g+1 \\
			\hline
			Nodo & |sk + 1| & |sk + 1|\\
			\hline
		\end{tabular}\\ \\ \\
		\textbf{FORMULE : } \\ \\
			\underline{NUMERO MINIMO DI NODI(IPmin)} \\ \\
			\centerline{{\Large {IP\textit{min} = ${1 + 2 \cdot \sum_{i=0}^{h-2}({g + 1})^i}$}}} \\ \\
			\underline{NUMERO MASSIMO DI NODI(IPmax)} \\ \\
			\centerline{{\Large {IP\textit{max} = ${\sum_{i=0}^{h-1}({2g + 1})^i}$}}} \\ \\
			\underline{ALTEZZA DI UN B-TREE} \\ \\
			NK = \textit{Numero di chiavi del B-TREE} \\ 
			NK\textit{min} = ${1 + g(IPmin - 1) = 2(g + 1)^{h-1}-1}$ \\
			NK\textit{max} = ${2g(IPmax) = (2g + 1)^{h}-1}$ \\ 
			h\textit{min} = ${\log_{2g+1}({NK + 1})}$ \\ 
			h\textit{max} = ${1 + \log_{g+1}{(\dfrac{NK + 1}{2})}}$ \\ \\
			\centerline{{\Large h\textit{min} ${\leq}$ h ${\leq}$ h\textit{max}}} \\ \\
			\underline{COSTO INSERIMENTO/ELIMINAZIONE DI UN NODO DA UN B-TREE(g)} \\ \\
			\begin{tabular}{|c|c|c|c|}
				\hline
				CASO : & MIGLIORE & PEGGIORE & MEDIA \\
				\hline
				Inserimento & ${h + 1}$ & ${3h + 1}$ & ${h + 1 + 2/g}$ \\
				\hline
				Eliminazione & ${h + 1}$ & ${3h}$ & ${5h + 5 + 3/g}$ \\
				\hline
			\end{tabular} \\ \\ \\
			\underline{STIMA VARIABILITA' DELL'ORDINE DI UN B-TREE} \\ \\
			\textit{k} = chiave \\
			\textit{p}  = \textit{RID} = Record Identifier, ovvero puntatore a record \\
			\textit{q} = \textit{PID} = Puntatore al nodo figlio \\
			\textit{D} = page size \\ \\
			${2g*len(k) + 2g*len(p) + (2g+1)*len(q) \leq D}$ \\ \\
			\centerline{{\Large g = ${\dfrac{D - len(q)}{2(len(k) + len(p) + len(q))}}$}} \\
	\end{titlepage}
	\begin{titlepage}
	\title{{\huge \textbf{FORMULE SUI B+-TREE}}}
	\maketitle\\\\
	\\\\
	DEFINIZIONE DI B+-TREE :	\\
	Un B+-TREE è un B-TREE in cui i record pointer sono memorizzati solo nei nodi foglia dell'albero. La struttura dei nodi foglia differisce quindi da quella dei nodi interni. 
	Ha prestazioni migliori del B-TREE sulla ricerca sequenziale, peggiori nella ricerca per singolo valore. \\ \\ \\
	\underline{ORDINE} \\ \\
	\textit{k} = chiave \\
	\textit{q} = \textit{PID} = Puntatore al nodo figlio \\
	\textit{D} = page size \\ \\
	${2g*len(k) + (2g+1)*len(q) \leq D}$ \\ \\
	\centerline{{\Large g = ${\dfrac{D - len(q)}{2(len(k) + len(q))}}$}} \\ \\ \\
	\underline{NUMERO DI FOGLIE} \\ \\
	\textit{NR} = Numero di Record \\
	\textit{NL} = Numero di foglie \\
	\textit{u} = \% di utilizzo di un singolo nodo (in media è il 69\%) \\
	\textit{d} = dimensione dei nodi \\ \\
	\centerline{{\Large NL = ${\dfrac{NR \cdot (len(k) \cdot len(q))}{d \cdot u}}$}} \\ \\
	\underline{ORDINE DELLE FOGLIE} \\ \\
	\textit{D} = page size \\ \\
	${2g_{leaf} \cdot len(k) + 2g_{leaf} \cdot len(p) + 2 \cdot len(q) \leq D}$ \\ \\
	\centerline{{\Large ${ g_{leaf} = \dfrac{D - 2 \cdot len(q)}{2 \cdot (len(k) + len(p))}}$}} \\ \\
	\textit{Con questo risultato si è in grado di avere una stima più accurata del numero di foglie di un B-+TREE} \\ \\
	\underline{NUMERO DI FOGLIE (stima più accurata sfruttando ${g_{leaf}}$)} \\ \\
	\centerline{{\Large NL = ${\dfrac{NR}{2g_{leaf} \cdot u}}$}} \\ \\
	\underline{ALTEZZA} \\ \\
	\textit{Costruendo un albero in cui ciascun nodo ha il numero massimo di figli, si minimizza l'altezza del B+-TREE} \\ \\
	${(2g + 1)^{h - 1} \ge NL}$ \\
	${h_{min} = 1 + \log_{2g+1}NL}$ \\ \\
	\textit{Similmente, sfruttando lo stesso ragionamento, costruendo un albero con il numero minimo possibile di figli per nodo, si massimizza l'altezza} \\ \\
	${2(g + 1)^{h - 2} \leq NL}$ \\ 
	${h_{max} = 2 + \log_{2g+1}\tfrac{NL}{2}}$ \\ \\
	\centerline{{\Large h\textit{min} ${\leq}$ h ${\leq}$ h\textit{max}}} \\ \\
	\underline{RICERCA DI VALORI} \\ \\
	\textit{Supponendo di avere un B-+TREE con NK chiavi in NL foglie, si effettua una ricerca sequenziale di k chiavi nell'intervallo di valori compresi fra [${k_{low}, k_{high}}$]} \\ \\
	${EK}$ = Numero di chiavi all'interno dell'intervallo [${k_{low}, k_{high}}$] \\
	${EL}$ = expected leafs, ovvero è una stima del numero di foglie alla quale si deve accedere durante la ricerca \\ \\
	${EL = \dfrac{EK \cdot NK}{NL}}$ (proporzione -> NK : NL = EK : EL) \\ \\
	\centerline{{\Large COSTO(Ricerca) = ${1 - h + EL}$}} \\ \\
	\underline{NUMERO DI FOGLIE DI UN SECONDARY B+-TREE} \\ \\
	\centerline{${NL = \dfrac{NK \cdot len(k) + NR \cdot len(p)}{D * u}}$}  \\ \\
	\underline{ALTEZZA DI UN SECONDARY B+-TREE} \\ \\
	${h_{min} = 1 + \log_{2g+1}min(NL,NK)}$ \\ \\
	${h_{max} = 2 + \log_{g+1} \dfrac{min(NL, NK)}{2}}$ \\ \\
	\centerline{{\Large h\textit{min} ${\leq}$ h ${\leq}$ h\textit{max}}} \\ \\
	\end{titlepage}
	\begin{titlepage}
	\title{{\huge \textbf{MODELLO DI CARDENAS}}}
	\maketitle\\\\
	\textit{Il modello di Cardenas è utile per stimare il numero medio di pagine NP che contengono almeno uno degli ER(expected records) presi in considerazione.} \\ \\
	Considerando i seguenti eventi : \\ \\
	${A}$ = "Una pagina contiene 1 degli ER record" \\
	${\Anegato}$ = "Una pagina \textbf{non} contiene 1 degli ER record" \\
	${B}$ = "Una pagina \textbf{non} contiene \textbf{nessuno} degli ER record " \\
	${\Bnegato}$ = "Una pagina contiene almeno 1 degli ER record" \\ 		
	Definiamo : \\ \\
	${P(A) = \dfrac{1}{NP} }$\space \space \space \space \space \space \space \space \space \space \space
	${P(\Anegato) = 1 - \dfrac{1}{NP}}$ \\ \\
	${P(B) = P(\Anegato)^{ER}}$ \space \space \space \space \space
	${P(\Bnegato) = 1 - P(B)}$ \\ \\
	\textbf{FORMULA DI CARDENAS :} \\ \\
	${\phi(ER,NP) = NP \cdot P(\Bnegato)}$ \\ \\
	${= NP * (1 - (1 - \dfrac{1}{NP})^{ER}) \space \leq \space MIN(ER, NP)}$
	\end{titlepage}
	\begin{titlepage}
	\title{{\huge \textbf{QUERY PLAN}}} \\ \\
	\maketitle
	Gestione del piano di accesso alle query, l'obiettivo e quello di scegliere il metodo più veloce ed efficiente. \\ \\
	\centerline{{\Large \textbf{Linear counting}}} \\ \\
	Algoritmo che permette di stimare il numero di NK chiavi distinte di un attributo A, con A $\in$ r. \\
	r = una qualsiasi estensione di una relazione R(T). \\ \\
	\textbf{DATI} : \\ \\
	BM = mappa chiave - valore \\
	H = hash function \\
	$t_{i}$.A = valore dell'elemento $\in$ A di una determinata tupla. \\
	B = n. totale di bucket della bitMap \\
	Z = n. di bucket della bitMap che hanno value = 0 \\ \\ 
	\textbf{Nota} : \\
	Ogni elemento $t_{i}$.A sarà presente all'interno della bit-map 
	BM seguendo la seguente logica: \\ \\
	if ( ${t_{i}.A}$ == NULL ); then BM[H(${t_{i}}$.A)] = 0; \\
	else then BM[H($t_{i}.A$))] = 1; \\ \\
	\centerline{\textbf{FORMULA RISOLUTIVA : }} \\
	\begin{center}
		${NK^{e} = -B*\ln{(\dfrac{Z}{B})}}$  \\
	\end{center}
	\end{titlepage}
	\textbf{PROCEDIMENTO} \\ \\
	Considerando l'evento A = "un bucket contiene ALMENO un valore". \\
	dal modello di cardenas si avrà :\\ \\
	${P(A) = (1 - (1 - (\dfrac{1}{B})^{NK}))}$ \\
	${P(\Anegato) = (1 - (\dfrac{1}{B})^{NK})}$ \\ \\
	il quale ci permetterà di formulare : \\ \\
	${B - Z = B \cdot P(A)}$ = \textit{stima del n. di bucket pieni} \\ \\
	${Z = B \cdot P(\Anegato)}$ = \textit{stima del n. di bucket vuoti} \\ \\
	Successivamente è importante eseguire la seguente approssimazione, applicabile solo nel caso in cui si lavori con valori molto grandi : \\ \\
	${P(\Anegato) \simeq e^{-\tfrac{NK}{B}}}$ = ${P(\modificato{\Anegato})}$ \\ \\
	di conseguenza Z diventerà : \\ \\
	${Z = B \cdot P(\modificato{\Anegato}) = B \cdot e^{-\tfrac{NK}{B}}}$ \\ \\
	A questo punto, è possibile estrarre NK, il valore cercato. \\ 
	\begin{center}
		${NK^{e} = -B*\ln{(\dfrac{Z}{B})}}$  \\ 
	\end{center} 
	l'apice \textit{e} aggiunto ad NK sta a significare estimated, ovvero il valore stimato, non necessariamente esatto, è stato aggiunto nel risultato per non confonderlo con la costante di Eulero. \\
	\begin{titlepage}
		\title{{\huge \textbf{Selettività dei predicati}}} \\ \\
		${f_{p}}$ = fattore di selettività; \space
		\maketitle
		${0 \leq f_{p} \leq 1}$ \\
		${ER = f_{p} \cdot NR = expectedRecords}$ \\
		da cui : \\
		${f_{p} = \tfrac{EK}{ER} = valoriSelezionati/valoriTotali}$ \\ \\
		\textbf{Casi notevoli} \\ \\
		\begin{tabular}{|c|c|}
			\hline 
			\textbf{Predicato} & \textbf{Formula } \\
			\hline \\
			= & ${f_{p} = \tfrac{1}{NK}}$ \\ \\
			\hline \\
			IN & ${f_{p} = \tfrac{card(Set)}{NK}}$\\ \\
			\hline \\ 
			< & ${f_{p} = \tfrac{v - min(R.A)}{max(R.A) - min(R.A)} \cdot \tfrac{NK - 1}{NK}}$ \\ \\
			\hline \\
			BETWEEN & ${f_{p} = \tfrac{v2 - v1}{max(R.A) - min(R.A)} \cdot \tfrac{NK - 1}{NK} + \tfrac{1}{NK}}$ \\
			\hline
		\end{tabular} \\ \\ \\
		\textbf{Predicati composti} \\ \\
		p = ${(p_{1}}$ AND ${p_{2})}$ $\hookrightarrow$ ${f_{p} = f_{p1} \cdot f_{p2}}$ \\ \\
		p = ${(p_{1}}$ OR ${p_{2})}$ $\hookrightarrow$ ${f_{p} = f_{p1} + f_{p2} - f_{p1} \cdot f_{p2}}$ \\ \\
		p = $\overline{p_{1}}$ $\hookrightarrow$ ${f_{p} = 1 - f_{p1}}$
		\begin{titlepage}
		\title{\huge \textbf{Costo di un query plan}} \\ \\
		\maketitle
		\textbf{full table scan} \\
		${C(SeqR) = C_{I/O}(SeqR) = NP \cdot \alpha \cdot NR}$ \\
		\textbf{Generale : IX(A) index scan} \\ 
		EN + EL + EP + $\alpha$ $\cdot$ ER \\ 
	\begin{center}
			\textbf{ \huge Scan con indice clustered} \\
	\end{center}
	\begin{center}
			\textbf{Su singolo valore o range:} \\
		EN = ${h - 1}$ \\
		EL = ${f_{p(A)} \cdot NL}$ \\
		EP = ${f_{p(A)} \cdot NP}$ \\
	\end{center}
	\begin{center}
		\textbf{Sul set :} \\ 
	EN = ${(h - 1) \cdot EK}$ \\
	EL = ${EK * \tfrac{NL}{NK}}$ \\
	EP = ${EK \cdot \tfrac{NP}{NK}}$ \\
	\end{center}
	\begin{center}
		\textbf{ \huge Scan con indice unclustered} \\
	\end{center}
		\begin{center}
			\textbf{Su singolo valore o range : } \\
		EN = ${h - 1}$ \\
		EL = ${f_{p} \cdot NL}$ \\
		EP = ${EK \cdot \phi(\tfrac{NR}{NK},NP)}$ \\
		\end{center} 
		\begin{center}
			\textbf{Su set : } \\
		EN = ${(h - 1) \cdot EK}$ \ \\
		EL = ${EK \cdot \tfrac{NL}{NK}}$ \\
		EP = ${EK \cdot \phi(\tfrac{NR}{NK},NP)}$ \\	
		\end{center}
		\end{titlepage}
		\begin{titlepage}
		\title{\huge \textbf{Proiezione}} \\ \\
		\maketitle
		SELECT DISTINCT <select list> FROM R \\
		Y = ${{(R.A_{1}, R.A_{2}, ..., R.A_{N})}}$ ( attributi citati nella select list) \\ \\
		\textbf{Modalità di calcolo di ER(Expected records)} \\ \\
		$ ER_{\pi_{Y(R)}} = NR_{R} $ (i vari $ NK_{R.A} $ non sono noti, si assume il caso peggiore) \\ \\
		$ ER_{\pi_{Y(R)}} = NK_{R.A} $(se la proiezione riguarda un solo attributo)  \\\\
		$ ER_{\pi_{Y(R)}} = min(NR_{r}, \prod_{i}^{} NK_{R.A_{i}}) $(caso peggiore : si assume indipendenza tra gli attributi che fanno parte di Y(non superchiave) \\\\
		\textbf{fattori di selettività della proiezione} \\ \\
		$ f_{\pi_{Y(R)}} = 1 $ (se la proiezione contiene una chiave candidata) \\ \\\\
		$ f_{\pi_{Y(R)}} = \dfrac{NK_{a}}{NR_{R}} $(se la proiezione riguarda un solo attributo A)\\ \\\\
		$ f_{\pi_{Y(R)}} = \dfrac{min(NR_{r}, \prod_{i}^{} NK_{R.A_{i}})}{NR_{R}} $(si assume indipendenza fra gli attributi) \\ \\
		\textbf{cardinalità di una proiezione multi attributo} \\ \\
		${\phi(NR, NK_{R.A} \cdot NK_{R.B})}$ \\ \\ \\ \\
		\textbf{Proiezione basata su sorting} \\ \\  
		1. Full scan di R e creazione della table T, che conterrà gli elementi selezionati dalla select(ResultSet). \\ \\
		C(seq R and build T) = $ NP_{R} + \alpha \cdot NR_{R} + EP_{T}  $ \\ \\
		2. Sia $EP_{T}$ il numero stimato di pagine di T e sia $NR_{T} = NR_{R} $ il numero di record di T.
		Si ordina T usando la combinazione lessicografica di tutti gli attributi. \\  \\
		$C(sort(EP_{T})) = C_{I/O}(sort(EP_{T})) + C_{CPU}(sort(EP_{T}))$ \\ \\
		$C_{I/O}(sort(EP_{T})) =  2 \cdot EP_{T} \cdot (1 + \log_{z}(\tfrac{EP_{T}}{(Z + 1) \cdot FS})) $ \\ \\
		3.Si scandisce T eliminando i duplicati : \\ \\
		C(seq T) = $ EP_ {T} + \alpha \cdot NR_{T} $
		\end{titlepage}
		\begin{titlepage}
		\title{\huge \textbf{Query di join}} \\ \\ 
		\maketitle
		\textbf{ Theta join} \\ \\
		Assumiamo di avere 2 relazioni, chiamate rispettivamente R ed S. \\ \\
		$f _{F_{J(R,S)}} = \dfrac{| R \Join S |}{| R \times S |}$ : fattore di selettività senza predicati locali \\ \\
		$f_{R, S}$ : fattore di selettività della query di join con predicati locali  \\ \\
		$ER_{q_{R, S}}(Expected Records from q_{r,s}) = f_{R, S} \times NR_{R} \times NR_{S} $ : numero di record risultanti dalla query $q_{R,S}$ \\ \\
		$ER_{F_{J(R, S)}} = f _{F_{J(R,S)}} \times NR_{R} \times NR_{S}$ : numero di record del risultato di puro join senza predicati locali \\ \\
		\textbf{Equi-join} \\ \\
		due relazioni legate da una condizione di uguaglianza \\ \\
		$f_{P(R.S = R.R)} = \dfrac{1}{max(NK_{R.S}, NK_{R.R})}$ \\ \\
		\textbf{caso 1 : equi-join su primary e foreign key} \\ \\
		con $NK_{R.J}$ e $NK_{S.J}$  cardinalità degli attributi R.J e S.J  \\ \\
		$f_{p(R.J = S.J)} = \dfrac{ER}{NR_{R} \times NR_{S}} = \dfrac{NR_{S}}{NR_{R} \times NR_{S}} = \dfrac{1}{NK_{R.J}} $ \\ \\
		$ER_{F_{J(R,S)}} = NR_{S}$ (R.J è PK mentre S.J è FK) \\ \\
		\textbf{caso 2 : equi-join m : n (la stima non è su PK ne su SK)} \\ \\
		$f_{p(R.J = S.J)} = \dfrac{1}{NK_{R.J}}$ \\ \\
		$ER_{F_{J(R, S)}} = f_{p(R.J = S.J)} \times NR_{R} \times NR_{S}$
		
		\end{titlepage}
	\end{titlepage}

\end{document}